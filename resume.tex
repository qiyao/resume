
\documentstyle[hyperref, margin, line]{res}

%\usepackage{hyperref} %previously in the documentstyle options
%options to hyperref package when in the documentstyle options
\hypersetup{backref, pdfpagemode=FullScreen, colorlinks=true,backref}

\def\Cplusplus{{\rm C\raise.5ex\hbox{\small ++}}}
% 'st' 'nd' 'rd' 'th' superscripts for numbers
\def\first{{\raise.5ex\hbox{\small st}}}
\def\second{{\raise.5ex\hbox{\small nd}}}
\def\third{{\raise.5ex\hbox{\small rd}}}
\def\fourth{{\raise.5ex\hbox{\small th}}}

%use most of page
%\addtolength{\oddsidemargin}{-0.5in}
%\addtolength{\voffset}{-0.5in}
%\addtolength{\textwidth}{1.25in}
%\addtolength{\textheight}{1in}

%use only a bit more page than default
\addtolength{\oddsidemargin}{-0.30in}
\addtolength{\voffset}{-0.50in}
%was -.30
\addtolength{\textwidth}{0.90in}
\addtolength{\textheight}{1.50in}
%was .60

%my name format
\renewcommand{\namefont}{\LARGE\emph}

\begin{document}
\name{Qi Yao}
\address{
  CodeSourcery
}
%\address{
%  Current Address:
%  15 Briarwood lane\\
%  Marlborough, MA 01752\\
%  USA
%}
%\address{
%  Permanent Address:
%  377 Rue De Aqueduc\\
%  Quebec, G1N 2N7\\
%  Canada
%}
\address{
  \href{yao@codesourcery.com}{\texttt{yao@codesourcery.com}}\\
}

\begin{resume}

\section{\textsc{PERSONAL PARTICULARS}}
\begin{tabular}{l l}
Name & \texttt{Qi Yao} \\
Sex & \texttt{Male} \\
%Date of Birth & \texttt{4 1 1982} \\
Place of Birth & \texttt{ShannXi Province} \\
Degree & \texttt{Master Degree} \\
Major & \texttt{Computer Science} \\
School & \texttt{Beijing Institute of Technology} \\
\end{tabular}
\section{\textsc{CONTACT}}
\begin{tabular}{l l}
E-mail: & \href{qiyaoltc@gmail.com}{\texttt{qiyaoltc@gmail.com}} \\
Tel: & \texttt{(86-10) 82350741} \\
Mobile: & \texttt{134-3931-1927} \\
\end{tabular}

\begin{format}
  \employer{l}\title{r}\\
  \location{l}\dates{r}\\
  \body\\
\end{format}

\section{\textsc{EXPERIENCE \& PROJECTS}}
\employer{\textbf{CodeSourcery}}
\title{\emph{Sourcerer}}
\location{Beijing}
\dates{\textbf{July.2010 - Present}}
\begin{position}
\begin{enumerate}
\item[] \emph{GDB} maintainer for tic6x port.
\item[] Advanced asynchronous remote notification framework added into \emph{GDB}.
\item[] New \emph{MI} notifications added into \emph{GDB}.
\item[] New debugging technology using agent in \emph{GDB}.
\item[] GNU toolchain development for \href{http://www.linaro.org}{\emph{http://www.linaro.org}} Toolchain Working Group (sponsored by ARM).
\item[] Various code size optimization and impromvent on \emph{Thumb-2} instructions.  Improve register renaming pass in \emph{GCC} to prefer to low register on ARM-v7.  Benchmark results show 0.3\% improvements on code size.
\item[] Fix various GCC and GDB bugs.
\item[] Communicate with open source toolchain community smoothly, and push changes into repository.
\end{enumerate}
\end{position}

\employer{\textbf{IBM China Development Lab}}
\title{\emph{Staff Software Engineer}}
\location{Beijing}
\dates{\textbf{July.2008 - June.2010}}
\begin{position}
\begin{enumerate}
\item[] \emph{Multicore SDK} design and integrate an all-in-one toolkit for parallel programs on multicore platform.  Released on \href{http://www.alphaworks.ibm.com/tech/msdk}{\emph{alphaWorks}}.  Some best algorithms and implementation methods are invented in order to resolve some challenging problems, like runtime overhead and accuracy.  It is the \emph{first} product tool for parallel java program analysis.
\item[] \emph{MultiThread Runtime Analysis Tool} design and development for multi-core platform.  Released on \href{http://www.alphaworks.ibm.com/tech/mtrat}{\emph{alphaWorks}}

\item[] \textbf{Qi, Yao} and Das, Raja and Luo, Zhi Da and Trotter, Martin.  MulticoreSDK: a practical and efficient data race detector for real-world applications.  PADTAD '09: Proceedings of the 7th Workshop on Parallel and Distributed Systems.
\item[] \textbf{Yao Qi}, Yarden Nir-buchbinder, Raja Das, Zhi Da Luo, Eitan Farchi and Zhi Gan.  Unit Testing for Concurrent Business Code.  PADTAD’10 July 12, Trento, Italy.
\item[] Zhi Da Luo, Linda Hillis, Raja Das and \textbf{Yao Qi}. Effective Static Analysis to Find Concurrency Bugs in Java.  2010 10th IEEE Working Conference on Source Code Analysis and Manipulation.
\item[] Zhi Da Luo, Raja Das, and \textbf{Yao Qi}.  MulticoreSDK : A Practical and Efficient Deadlock Detector for Real-World Applications.  2011 The IEEE International Conference on Software Testing, Verification and Validation.
\end{enumerate}
\end{position}

\employer{\textbf{IBM China Development Lab}}
\title{\emph{Software Engineer}}
\location{Beijing}
\dates{\textbf{July.2006 - July.2008}}
\begin{position}
\begin{enumerate}
\item[] \href{http://www.sourceware.org/frysk}{\emph{frysk}} development and enablement for \emph{PowerPC64} platform.
\item[] \href{http://www.sourceware.org/gdb}{\emph{GDB}} development and \emph{checkpoint} implementation.
\item[] Contributor of \emph{Decimal Float Point} in \href{http://gcc.gnu.org}{\emph{GCC}}.
\item[] Design and implement user-mode instruction trace on \href{http://www.valgrind.org/}{\emph{Valgrind}}.
\item[] Write testsuites for \emph{Oprofile}, \emph{ltrace}, and \emph{Strace}.
\end{enumerate}
\end{position}


\section{\textsc{Professional Skills}}
\begin{enumerate} 
\item[] Deep knowledge of \emph{compiler}, \emph{debugger}, \emph{thread mechanism} and \emph{system run-time} features.
\item[] Deep understand of \emph{Java Virtual Machine}, \emph{Byte Code Instrumentation} and \emph{Just In Time} compilation.
\item[] Proficient in \emph{C}, \emph{C++}, \emph{JAVA}.
\item[] Experienced on \emph{embedded system}, and computer architecture(\emph{ARM} and \emph{PowerPC}).
\item[] Experienced on development on \emph{UNIX} platforms, such as Solaris and AIX.
\item[] Experienced on communication and work model in open source community.
\item[] Familiar with \emph{Shell script}, \emph{Perl}, and \emph{Python}.
\item[] Daily life in \emph{Linux}.
\item[] Good at \emph{Tcl/Expect} and 
 \LaTeX
\end{enumerate}

\section{\textsc{Technical Achievements}}
\begin{enumerate}
\item[] A good experience on design and implement a dynamic and static analysis tool for concurrent java program with low overhead and high accuracy.
\item[] A active innovator.  Certified IBM \emph{Senior Inventor}(the first five in Great China Group).  Filed more than 20 patents.
\item[] A good open source developer.  Familiar with open source development model.
\item[] Good communication in English.
\item[] Good leadership.  Co-chair of \emph{Invention Development Team} in IBM China Software Development Lab
\end{enumerate}

\section{\textsc{EDUCATION}}

\employer{\textbf{Computer Science}}
\title{\emph{Master}}
\location{Beijing Institute of Technology}
\dates{\textbf{Sep.2004 - July.2006}}
\begin{position}
\end{position}

\employer{\textbf{Computer Science}}
\title{\emph{Bachelor}}
\location{Beijing Institute of Technology}
\dates{\textbf{Sep.2000-Jul.2004}}
\begin{position}
\end{position}


%\section{\textsc{International Skills}}
%\emph{Languages}: Spoken and written English; spoken Swiss German and some French

%\emph{Citizenship}: Canada, Switzerland, France


%Association for Computing Machinery.
%National and Worcester chapter member.\\
%1997 -- Present

%; GPA: 3.1

%\section{\textsc{Course Work}}
%  \begin{tabular}{lll}
%  Operating Systems I \& II	&
%  Logic Circuits		&
%  Computer Graphics		\\
%  Introduction to AI		&
%  Microprocessor Systems	&
%  Computer Music		\\
%  Computer Architecture		&
%  Software Engineering		&
%  Algorithms			\\
%  Compilers 			&
%  Computer Networks 		&
%  Human-Computer Interaction	\\
%  \end{tabular}

%\section{\textsc{Project Work}}
%\emph{Major Qualifying Project} --- An Iconic Recipe Server.\newline
%	Design of a system for non-verbal [iconic] representation of sets of
%	instructions [recipes] coupled with a networked Java based client \&
%	server for the entry and retrieval of iconic recipes.

%\emph{Interdisciplinary Qualifying Project} --- The Digital Peddler.\newline
%	Production of a digital yearbook; a study of digital multimedia
%	production as opposed to traditional media production.

%\emph{Humanities Sufficiency Project} --- Continuity and the Self.  A topic in
%Philosophy.


%
%November 19th, 2001
%
\end{resume}
\end{document}
